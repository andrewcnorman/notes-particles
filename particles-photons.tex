\section{Particles and their antiparticles}

Every particle that exists has an antiparticle.  Antiparticles do not exist as constituents of ordinary matter, but are easily created in particle accelerators, and are produced in radioactive decay and cosmic rays.

For every particle:
\begin{itemize}
\item its antiparticle has the same mass
\item its antiparticle has equal but opposite charge (as well as other associated numbers)
\item an unstable particle has the same half-life as its antiparticle
\end{itemize}

\subsection{Stable particles}

The following particles are the only particles which are stable:\\

\begin{table}[ht]
  \centering
  \fontfamily{ppl}\selectfont
  \begin{tabular}{lcccc}
    \toprule
    Name & Symbol & Charge ($Q$) / $e$ & Rest mass / kg & Rest energy / MeV\\
    \midrule
    electron & \Pelectron & $-1$ & 9.1\e{-31} & 0.511\\
    positron & \Ppositron & $+1$ & 9.1\e{-31} & 0.511\\
    proton & \Pproton & $+1$ & 1.6\e{-27} & 938\\
    antiproton & \APproton & $-1$ & 1.6\e{-27} & 938\\
    neutrino & \Pnu & 0 & $\sim 10^{-37}??$ & $<\num{18.2e-6}$\\
    antineutrino & \APnu & 0 & $\sim 10^{-37}??$ & $<\num{18.2e-6}$\\
    \bottomrule
  \end{tabular}
%  \caption{The particles which make up matter and thei}
\end{table}

\paragraph{Notes}\begin{enumerate}
\item The electron volt (eV) is a unit of {\bf energy}.\footnote{The electron volt is the amount of energy gained by an electron when it is accelerated through a potential difference of 1 volt.}  In atomic and nuclear physics, the joule is too large a unit for convenience.
\[\SI{1}{eV}=\SI{1.6e-19}{J}\text{ and }\SI{1}{MeV}=\SI{e6}{eV}.\]
\item The antiparticles are stable in isolation: in practice, they would likely encounter a particle and annihilate if they came into contact with normal matter.
\item There are three kinds of neutrino, and therefore three antineutrinos.\footnote{In fact, neutrinos are constantly shifting between these three flavours of neutrino -- this is how physicists know they have mass!}
\item Some particles are identical to their anti-particle, e.g.\ the photon, and the \Ppizero (pi-meson).
\item In general, particle symbols are made into their antiparticle by the `bar' above the symbol.
\end{enumerate}

\subsection{Neutrinos}
The neutrino was predicted to exist in 1930, and was postulated due to the range of energies of beta particles in radioactive beta decay.

Neutrinos are the most common particles in the universe, outnumbering the number of protons by about a billion to one.  They are emitted in nuclear reactions such as those that occur inside the sun.

They are very difficult to detect, even though 6\e{10} pass through every square centimetre of the earth every second.  They interact very weakly, and this is why their mass was not discovered until recently, and is an area of ongoing research.

\subsection{Photons}

Light, and all of the electromagnetic spectrum, can be described as a wave, and this adequately explains effects such as diffraction and reflexion.  However, physicists found that some phenomena, such as the photoelectric effect\footnote{This is discussed in more detail in `quantum phenomena'.  Briefly, it is the emission of electrons from the surface of a metal.  It is noted that (i) there is a minimum frequency of light to cause emission (the electrons need a minimum energy to escape) and (ii) the kinetic energy of the electrons emitted increases as the frequency of the light increases (photons have more energy to give to the electrons).} and the black body radiation spectrum\footnote{You don't have to know about this at A-level.}, could not be described in this way.

In 1900, Max Planck came up with a suggestion that objects which emit electromagnetic radiation do so in discrete amounts.  He said that the energy was proportional to the frequency of the radiation, i.e.
\[E\propto f,\mathrm{~or}\]
\[E=hf,\]
where $h$ is the Planck constant, \SI{6.64e-34}{J.s}.

The individual packets, or quanta, of electromagnetic radiation are called photons.  Photons are indivisible, and when they collide with e.g.\ an electron, all or none of the energy is given to the electron.

\paragraph{Examples}

\begin{fullwidth}
\begin{enumerate}
\item Calculate the photon energy of
\begin{enumerate}
\item a gamma ray of frequency 2\e{22}~Hz\\
$E=hf=6.64\e{-34}~\mathrm{J~s~}\times 2\e{22}~\mathrm{s}^{-1}=1.33\e{-11}$~J.
\item red light of wavelength 7.8\e{-7}~m\\
$E=hf$, as $c=f\lambda$ (where $c=3\e{8}$~m~s$^{-1}$)\\
$E=\dfrac{hc}{\lambda}=\dfrac{6.64\e{-34}~\mathrm{J~s~}\times 3\e{8}~\mathrm{m~s}^{-1}}{7.8\e{-7}~\mathrm{m}}=2.55\e{-19}$~J.
\end{enumerate}
\item A sodium light emits yellow light of frequency 5.1\e{14}~Hz.  If it is a 30~W lamp, how many photons are emitted per second?\\
$30~\mathrm{W}=30$~J~s$^{-1}.$\\
$P=\dfrac{E}{t}=\mathrm{photon~number~per~unit~time}\times hf,$\\
$\mathrm{photon~number~per~unit~time}=\dfrac{P}{hf}=\dfrac{30~\mathrm{J}~\mathrm{s}^{-1}}{6.64\e{-34}~\mathrm{J~s~}\times 5.1\e{14}~\mathrm{Hz}}=8.9\e{19}~\mathrm{s}^{-1}.$
\end{enumerate}
\end{fullwidth}
