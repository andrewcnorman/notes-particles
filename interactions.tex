\section{Particle Interactions}

An interaction describes a collision between two particles resulting in new particles being formed, or a decay of an unstable particle into other particles.  

\subsection{Collisions}

When particles collide, the total energy that they contain is their kinetic energy and the energy that they have due to their mass.  After the collision, the particles produced may have different masses and kinetic energies, but the total must remain the same (conservation of energy).  This means that we can produce different particles by collisions.

e.g.\[\Pproton+\Pproton\longrightarrow\Pproton+\Pneutron+\Ppiplus\]

The collision of two protons gives a proton a neutron and a pi-plus.  Some of the kinetic energy of the protons goes into producing the extra mass of the neutron and pion.

\subsection{Decays}

Most particles are unstable, which means that they decay into other particles (similar to radioactive decay).  A particle will decay into other particles as long as the total mass of the products is less than the original particle, so that any excess mass will go into the kinetic energy of the products.

e.g.\ \[\Ppiplus\longrightarrow\APmuon+\Pnu_{\mu}\]
mass of $\pi^{+} = 2.5\e{-28}$~kg\\
mass of $\mu^{+} = 1.9\e{-28}$~kg\\
mass of $\nu_{\mu}$ is almost zero ($<3.4\e{-34}$~kg)\\


All unstable particles have a characteristic lifetime, which is the average time that it will take for that particle to decay.  

\subsection{Conservation laws}

In every interaction:
\begin{enumerate}
\item mass/energy must be conserved
\item momentum must be conserved
\item charge $Q$ must be conserved
\item baryon  number $B$ must be conserved
\item lepton number $L$ must be conserved
\item strangeness $S$ \begin{enumerate}\item is conserved in collisions
\item changes by $\pm 1$ in a weak decay\end{enumerate}
\end{enumerate}

\paragraph{Notes}
\begin{itemize}
\item Strange particles decay by strong or weak decays, and these can usually be distinguished by the lifetime of the decay (the typical lifetimes for strong decays are typically $10^{-23}$~s, and weak decays typically $10^{-8}$~s).  Strangeness may be `lost' in a weak decay: no strange particles are stable, and when one decays, the strangeness can change by $\pm 1$, so that the total strangeness gets closer to zero.  Although strangeness can change in this way in weak decays, in strong decays of strange particles, strangeness {\bf is} conserved. We shall presume all strange decays to be weak, i.e.\ strangeness will change if a strange particle decays, as a strange quark changes into a non-strange quark.
\item Energy/mass considerations will not be taken into account, as in principle any energy can be converted into mass in accelerators.
\end{itemize}

The general method of solution is to write $Q$, $B$, $L$, $S$ below the interaction and check for conservation.

\subsubsection{Examples}

\begin{enumerate}
\item Which of the following are possible?\begin{enumerate}
\item \begin{tabular}[t]{lccccccc}
& \Pneutron & $\longrightarrow$ & \Pproton & $+$ & \Pelectron & $+$ & $\APnu_{\mathrm{e}}$\\
$Q$ & $0$ & $\rightarrow$ & $+1$ & & $-1$ & & $0$ \\
$B$ & $+1$ & $\rightarrow$ & $+1$ & & $0$ & & $0$ \\
$L$ & $0$ & $\rightarrow$ & $0$ & & $+1$ & & $-1$ \\
$S$ & $0$ & $\rightarrow$ & $0$ & & $0$ & & $0$ \\
\end{tabular}\\
All of the quantities are conserved, so this $\beta$ decay is possible.
\item \begin{tabular}[t]{lccccc}
& $\PLambda^{0}$ & $\longrightarrow$ & \Pproton & $+$ & \Ppiminus \\
$Q$ & $0$ & $\rightarrow$ & $+1$ & & $-1$ \\
$B$ & $+1$ & $\rightarrow$ & $+1$ & & $0$  \\
$L$ & $0$ & $\rightarrow$ & $0$ & & $0$  \\
$S$ & $-1$ & $\rightarrow$ & $0$ & & $0$ \\
\end{tabular}\\
$Q$, $B$ and $L$ are conserved, and the strangeness changes by $+1$ in this weak decay.
\end{enumerate}
\item Identify particle X:\\
\begin{tabular}[t]{lccccccccccc}
& \Pproton & $+$ & \Ppiminus & $\longrightarrow$ & \Pneutron & $+$ & \Ppizero & $+$ & \Ppiminus & $+$ & X\\
$Q$ & $+1$ & & $-1$ & $\rightarrow$ & $0$ & & $0$ & & $-1$ & & $+1$ \\
$B$ & $+1$ & & $0$ & $\rightarrow$ & $+1$ & & $0$ & & $0$ & & $0$ \\
$L$ & $0$ & & $0$ & $\rightarrow$ & $0$ & & $0$ & & $0$ & & $0$ \\
$S$ & $0$ & & $0$ & $\rightarrow$ & $0$ & & $0$ & & $0$ & & $0$ \\
\end{tabular}\\
From its properties, the particle X must be a \Ppiplus .
\end{enumerate}
