\section{Classification of particles}
All fundamental particles fall into one of three families: the leptons, the quarks, and the gauge bosons.

\subsection{Leptons}

There are twelve different leptons, which do not experience the strong interaction (and uncharged leptons do not feel the electromagnetic interaction either).  They are the electron, the muon and the tau, each has an associated neutrino, and they all have antiparticles.  The standard model, a collection of related gauge theories currently used in particle physics, suggests that these are the only leptons which exist.  The electron and positron are stable.

The muon (\Pmuon) was discovered in cosmic rays in 1937, and has the same charge as the electron, but is 207 times heavier.  The neutrino which accompanies its reactions was found to be different from the neutrino in electron reactions in 1962.

The tau (\Ptauon) was discovered in 1978, and is also the same charge as the electron, but is around 3500 times more massive.  The neutrino that accompanies the tau was only discovered in 2000 (leaving the Higgs boson as the last particle in the standard model to be observed in 2012), though both particles' existence was assumed in the standard model long before this.

Both the muon and the tau decay into electrons.

\begin{table}
  \centering
  \fontfamily{ppl}\small\selectfont
  \begin{tabular}{lccccc}
    \toprule
    Name & Symbol & Charge ($Q$) / $e$ & \multicolumn{3}{c}{Lepton numbers}\\
    &&&$L_{\mathrm{e}}$&$L_{\mu}$&$L_{\tau}$\\
    %\cline{4-6}
    \midrule
    electron & $\Pelectron$ & $-1$ & $1$ & $0$ & $0$\\
    positron & $\Ppositron$ & $+1$ & $-1$ & $0$ & $0$\\
    electron neutrino & $\Pnu_{\mathrm{e}}$ & $0$ & $1$ & $0$ & $0$\\
    antielectron neutrino & $\APnu_{\mathrm{e}}$ & $0$ & $-1$ & $0$ & $0$\\
    \midrule
    muon & $\Pmuon$ & $-1$ & $0$ & $1$ & $0$\\
    antimuon & $\APmuon$ & $+1$ & $0$ & $-1$ & $0$\\
    muon neutrino & $\Pnu_{\mu}$ & $0$ & $0$ & $1$ & $0$\\
    antimuon neutrino & $\APnu_{\mu}$ & $0$ & $0$ & $-1$ & $0$\\
    \midrule
    tau & $\Ptauon$ & $-1$ & $0$ & $0$ & $1$\\
    antitau & $\APtauon$ & $+1$ & $0$ & $0$ & $-1$\\
    tau neutrino & $\Pnu_{\tau}$ & $0$ & $0$ & $0$ & $1$\\
    antitau neutrino & $\APnu_{\tau}$ & $0$ & $0$ & $0$ & $-1$\\
    \bottomrule
  \end{tabular}
  \caption{To help identify which reactions may take place, all particles are assigned a series of numbers.}
\end{table}

\paragraph{Notes}\begin{enumerate}
\item Baryon number $B$ and strangeness $S$ are zero for all leptons.
\item Only leptons have lepton numbers, and lepton number must be conserved (remain unchanged) in all particle interactions or decays.
\end{enumerate}

\subsection{Quarks}
Although quarks are fundamental particles, they \emph{never} exist in isolation, as single, free quarks.  The standard model suggests there should be six `flavours' of quark (i.e.\ six quarks and six antiquarks, which match the twelve leptons).  These all experience all of the four interactions, and always combine to form heavier particles called hadrons.

\begin{table}
  \centering
  \fontfamily{ppl}\small\selectfont
  \renewcommand{\arraystretch}{1.2}
  \begin{tabular}{lccc}
    \toprule
    Quark & Symbol & Charge ($Q$) / $e$ & mass / $m_{\mathrm{p}}$\\
    \midrule
    up & \Pup & $+\frac{2}{3}$ & 0.33\\
    antiup & \APup & $-\frac{2}{3}$ & 0.33\\ 
    down & \Pdown & $-\frac{1}{3}$ & 0.34\\
    antidown & \APdown & $+\frac{1}{3}$ & 0.34\\
    \midrule
    charm & \Pcharm & $+\frac{2}{3}$ & 1.59\\
    anticharm & \APcharm & $-\frac{2}{3}$ & 1.59\\
    strange & \Pstrange & $-\frac{1}{3}$ & 0.53\\
    antistrange & \APstrange & $+\frac{1}{3}$ & 0.53\\
    \midrule
    top & \Ptop & $+\frac{2}{3}$ & 185.5\\
    antitop & \APtop & $-\frac{2}{3}$ & 185.5\\
    bottom & \Pbottom & $-\frac{1}{3}$ & 4.80\\
    antibottom & \APbottom & $+\frac{1}{3}$ & 4.80\\
    \bottomrule
  \end{tabular}\\
  \renewcommand{\arraystretch}{1}
  \caption{The six quarks.  Note that we shall only consider combinations of up, down and strange quarks (and their antiquarks), although the other quarks follow the same patterns.  However, it is more difficult to make the heavier quarks as much more energy is required.}
\end{table}

The particles that quarks make up (called hadrons) fall into two categories, baryons (made up of three quarks) and mesons (made up of two quarks).

\subsection{Gauge bosons}

A gauge boson is a force carrier that carries one of the fundamental interactions of nature.  The Standard Model of particle physics has three kinds of gauge bosons: photons (which carry the electromagnetic interaction), W and Z bosons (which carry the weak interaction) and gluons (which carry the strong interaction).\footnote{The Higgs boson is not a gauge boson, as it arises in the standard model via a different mechanism, responsible not for an interaction but for the mass, e.g.\ of the W and Z bosons.  The graviton is not in the standard model, and has not been experimentally verified, but if it exists it might be a gauge boson.}

