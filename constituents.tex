
\section{Constituents of the atom}

The atom comprises a tiny ($\approx \SI{e-14}{m}$) nucleus, containing protons and neutrons, around which are electrons in atomic orbitals (of radius $\approx \SI{e-10}{m}$).

\begin{table}[ht]
  \centering
  \fontfamily{ppl}\small\selectfont
  \begin{tabular}{lcccc}
    \toprule
    Name & Location & Charge / C & Relative mass & Actual mass / kg\\
    \midrule
    Proton & nucleus & +\num{1.6e-19} & 1 & \num{1.67e-27}\\
    Neutron & nucleus & 0 & 1 & \num{1.67e-27}\\
    Electron & orbitals & $-$\num{1.6e-19} & $1/1833$ & \num{9.11e-31}\\
    \bottomrule
  \end{tabular}
  \caption{The particles which make up matter and their properties.}
\end{table}

An atom is written as

\begin{center}
{\huge \isotope[{\it A}][{\it Z}]{X}}
\end{center}

where \begin{minipage}[t]{12cm}$A$ is the nucleon number (the number of protons and neutrons),\\
$Z$ is the proton number, and\\
X is the element symbol.\end{minipage}

\subsection{Proton number, $Z$}

Also called the atomic number.  This defines the element, and therefore dictates its properties.  In an atom, the number of electrons will equal the proton number; in an ion, there will be fewer or more electrons than $Z$.

\subsection{Nucleon number, $A$}

Also called the mass number.  This is the total number of nucleons (i.e.\ protons + neutrons) in the nucleus.

The number of neutrons is therefore $A-Z$.  All nuclei, except for one isotope of hydrogen, contain neutrons.\footnote{In general, for lower $Z$ elements, there are roughly the same numbers of protons and neutrons, but the number of neutrons increases more rapidly as large nuclei are made.}  The neutrons hold together the protons, which electrostatically repel each other.

The number of neutrons have no effect on the chemical properties of the element, but may make it more or less stable and therefore determine whether an element is radioactive.

\subsection{Isotopes}

Isotopes are nuclides with the same proton number, but different nucleon numbers (i.e.\ same number of protons, but different numbers of neutrons).

Many elements exist in several stable isotopes, and they are not given separate names, except for:
\begin{itemize}
\item \isotope[1][1]{H} is hydrogen.
\item \isotope[2][1]{H} is deuterium.
\item \isotope[3][1]{H} is tritium.
\end{itemize}
