\subsection{Baryons ($\Pquark\Pquark\Pquark$ or $\APquark\APquark\APquark$)}

Baryons have a baryon number $B$ of $+1$, are made of three quarks, and have a lepton number $L$ of $0$.  They may or may not have a strangeness, which depends on whether they contain strange quarks. (Antibaryons contain three antiquarks, and have a baryon number of $-1$.)

\begin{table}
  \centering
  \fontfamily{ppl}\small\selectfont
  \begin{tabular}{lcccc}
    \toprule
    Name & Symbol & Charge ($Q$) & Baryon No. ($B$) & Strangeness ($S$)\\
    \midrule
    proton & \Pproton & $+1$ & $1$ & $0$ \\
    antiproton & \APproton & $-1$ & $-1$ & $0$ \\
    neutron & \Pneutron & $0$ & $1$ & $0$ \\
    antineutron & \APneutron & $0$ & $-1$ & $0$ \\
    lambda & \PLambda & $0$ & $1$ & $-1$ \\
    antilambda & \APLambda & $0$ & $-1$ & $+1$ \\
    sigma plus & \PSigmaplus & $+1$ & $1$ & $-1$ \\
    antisigma plus & \APSigmaplus & $-1$ & $-1$ & $+1$ \\
    sigma zero & \PSigmazero & $0$ & $1$ & $-1$ \\
    antisigma zero & \APSigmazero & $0$ & $-1$ & $+1$ \\
    sigma minus & \PSigmaminus & $-1$ & $1$ & $-1$ \\
    antisigma minus & \APSigmaminus & $+1$ & $-1$ & $+1$ \\
    xi minus & \PXiminus & $-1$ & $1$ & $-2$ \\
    antixi minus & \APXiminus & $+1$ & $-1$ & $+2$ \\
    xi zero & \PXizero & $0$ & $1$ & $-2$ \\
    antixi zero & \APXizero & $0$ & $-1$ & $+2$ \\
    omega & \POmega & $-1$ & $1$ & $-3$ \\
    antiomega & \APOmega & $+1$ & $-1$ & $+3$ \\
    \bottomrule
  \end{tabular}
\caption{There are many more baryons than the combinations of the six quark flavours suggests ($^{6}C_{3}\times 2=40$), as the quarks have energy levels (like electrons).  The $\Delta^{+}$ (delta plus) particle, for example, has the same quark structure as the proton, but is more massive as its quarks have more energy, and spin differently.}
\end{table}

Of all the baryons, only the proton is stable (and the standard model suggests its half life may be around $10^{32}$~years, cf. present age of the universe $\sim 10^{10}$~years).  All other baryons are unstable and will eventually decay into protons (e.g.\ free neutrons which are not bound up in nuclear matter\footnote{When bound up in a nucleus, neutron instability via beta decay is balanced against the instability of the nucleus as a whole from the coulomb repulsion of the resulting additional proton, so bound neutrons are not necessarily unstable\ldots} will decay to protons with a half-life of about 10 minutes, 14 seconds).

\newpage

\subsection{Mesons ($\Pquark\APquark$)}

Mesons have a baryon number of zero, and a lepton number of zero.  They are made up of a quark and an antiquark.  None of the mesons are stable.

\begin{table}
  \centering
  \fontfamily{ppl}\small\selectfont
  \begin{tabular}{lcccc}
    \toprule
    Name & Symbol & Charge ($Q$) & Baryon No. ($B$) & Strangeness ($S$)\\
    \midrule
        {\bf Pions} & & & &\\
        pi-zero & \Ppizero & $0$ & $0$ & $0$ \\
        antipi-zero & \Ppizero & $0$ & $0$ & $0$ \\
        pi-plus & \Ppiplus & $+1$ & $0$ & $0$ \\
        pi-minus* & \Ppiminus & $-1$ & $0$ & $0$ \\
        \midrule
            {\bf Kaons} & & \\
            K-zero & \PKzero & $0$ & $0$ & $1$ \\
            antiK-zero & \APKzero & $0$ & $0$ & $-1$ \\
            K-plus & \PKplus & $+1$ & $0$ & $1$ \\
            K-minus* & \PKminus & $-1$ & $0$ & $-1$ \\
            \bottomrule
  \end{tabular}
\caption{*Due to their composition, the antiparticles of the pi-plus and the K-plus are not the anti-pi-plus and anti-K-plus (as might be expected by looking at the table of baryons), but are instead the pi-minus and K-minus.  There are many more mesons than those listed!}
\end{table}

\section{Quark structure of hadrons}

In the early 1960s, the number of known hadrons was large enough for physicists to start to make patterns and reduce the large numbers to a simpler scheme.  Murray Gell-Mann and George Zweig put together the theory that the hadrons were made up of smaller constituents named quarks.\footnote{Experimental evidence for quarks was first provided in 1969 when electrons of de Broglie wavelength \SI{e-16}{m} were fired at protons.  As this is about 10 times smaller than the proton, these electrons can resolve internal structure, finding three particles inside the proton.}

\subsection{Quark combinations}

\begin{enumerate}
\item Baryons are composed of three quarks ($\Pquark\Pquark\Pquark$) and antibaryons are composed of three antiquarks ($\APquark\APquark\APquark$).
\item Mesons are composed of a quark and an antiquark ($\Pquark\APquark$).
\end{enumerate}

\begin{table}
  \centering
  \fontfamily{ppl}\small\selectfont
  \renewcommand{\arraystretch}{1.2}
  \begin{tabular}{lccc}
    \toprule
    Quark & Charge ($Q$) / $e$ & Baryon number ($B$) & Strangeness ($S$) \\
    \midrule
    up (\Pup) & $+\frac{2}{3}$ & $+\frac{1}{3}$ & $0$ \\
    antiup (\APup) & $-\frac{2}{3}$ & $-\frac{1}{3}$ & $0$ \\ 
    down (\Pdown) & $-\frac{1}{3}$ & $+\frac{1}{3}$ & $0$ \\
    antidown (\APdown) & $+\frac{1}{3}$ & $-\frac{1}{3}$ & $0$ \\
    strange (\Pstrange) & $-\frac{1}{3}$ & $+\frac{1}{3}$ & $-1$ \\
    antistrange (\APstrange) & $+\frac{1}{3}$ & $-\frac{1}{3}$ & $+1$ \\
    \bottomrule
  \end{tabular}
  \renewcommand{\arraystretch}{1}
  \caption{It was initially thought that there were only three flavours of quark, up, down and strange.  Current theory suggests there are six.  The properties of the first three quarks to be discovered are shown (their lepton numbers are all zero).}
\end{table}

\begin{table}
  \centering
  \fontfamily{ppl}\small\selectfont
  \begin{tabular}{lclc}
    \toprule
    Particle & Quark content & Antiparticle & Quark content \\
    \midrule
    \Pproton & $\Pup\Pup\Pdown$ & \APproton & $\APup\APup\APdown$\\
    \Pneutron & $\Pup\Pdown\Pdown$ & \APneutron & $\APup\APdown\APdown$\\
    \Ppiplus & $\Pup\APdown$ & \Ppiminus & $\APup\Pdown$\\
    \Ppizero & $\Pup\APup$, $\Pdown\APdown$ & \Ppizero & $\Pup\APup$, $\Pdown\APdown$\\
    \PKplus & $\Pup\APstrange$ & \PKminus & $\APup\Pstrange$\\
    \PKzero & $\Pdown\APstrange$ & \APKzero & $\APdown\Pstrange$\\
    \bottomrule
  \end{tabular}\\
  \caption{The actual quarks involved in a particular particle can be calculated from the quantum numbers of that particle, although the proton and neutron should be known.}
\end{table}

The \Ppizero is actually a combination of $\Pup\APup+\Pdown\APdown$, but can be regarded as just one of these.  When a meson is made of a quark-antiquark pair in this way (e.g.\ a \Pup and \APup, which appear to be antiparticles) these quarks do not annihilate, as they have another property (colour\footnote{Colour is just a label---not a real colour!---for the charge of the strong interaction.  Unlike electric charge ($+$ or $-$), the strong charge comes in three colours: R, G and B.}) which is different for the two quarks.  This means that the \Pup and \APup in a \Ppizero, having different colours, do not annihilate.
