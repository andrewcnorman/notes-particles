
\section{Feynman diagrams}

When the American physicist Richard Feynman wanted to calculate the probability of interactions occurring, he drew a set of diagrams to show all possible outcome.  These apparently simple diagrams allow very complex calculations to be solved easily.  

Feynman diagrams represent particle interactions -- the angles between the particle lines are not significant, only the sequence of events.  The interaction itself is shown via an exchange particle.

\subsection{$\beta^{-}$ decay}
A neutron decays into a proton (a down quark changes into an up quark), emitting a beta particle and an antielectron neutrino:
\[\mathrm{n}\longrightarrow\mathrm{p}\mathrm{e}^{-}\bar{\nu}_{\mathrm{e}}\]

\noindent \begin{minipage}{0.4\textwidth}\begin{tikzpicture}
\draw[style={boson}]
  (0, 0) -- node[below] {$\mathrm{W}^{-}$} (2,0) ;
   \draw[->] (0.5,-0.6)--(1.5,-0.6);
    \draw[style={electron}]
   (-1, -2) -- node[left] {n}(0, 0);
  \draw[style={electron}]
  (0, 0) -- node[left] {p}(-1, 2) ;
  \draw[style={electron}]
   (2, 0)  -- node[right] {e$^{-}$}(1, 2) ;
  \draw[style={electron}]
  (2, 0) -- node[right] {$\bar{\nu}_{\mathrm{e}}$}(3, 2) ;
\end{tikzpicture} \end{minipage}\begin{minipage}[c]{0.1\textwidth}OR\end{minipage}
\begin{minipage}{0.4\textwidth}
\begin{tikzpicture}
\draw[style={boson}]
  (0, 0) -- node[below] {$\mathrm{W}^{-}$} (2,0) ;
   \draw[->] (0.5,-0.6)--(1.5,-0.6);
    \draw[style={electron}]
   (-1, -2) node[below] {d}-- (0, 0);
  \draw[style={electron}]
  (0, 0)  -- (-1, 2) node[above] {u};
  \draw[style={electron}]
   (-1.2, -2) node[below] {d}-- (-0.2, 0);
  \draw[style={electron}]
  (-0.2, 0)  -- (-1.2, 2) node[above] {d};
  \draw[style={electron}]
   (-1.4, -2) node[below=2.5pt] {u}-- (-0.4, 0);
  \draw[style={electron}]
  (-0.4, 0)  -- (-1.4, 2) node[above] {u};
  \draw[style={electron}]
   (2, 0)  -- node[right] {e$^{-}$}(1, 2) ;
  \draw[style={electron}]
  (2, 0) -- node[right] {$\bar{\nu}_{\mathrm{e}}$}(3, 2) ;
\end{tikzpicture}  
\end{minipage}

\subsection{$\beta^{+}$ decay}
A proton decays into a neutron, emitting a neutrino and positron:
\[\mathrm{p}\longrightarrow\mathrm{n}\mathrm{e}^{+}\nu_{\mathrm{e}}\]

\noindent \begin{minipage}{0.4\textwidth}\begin{tikzpicture}
\draw[style={boson}]
  (0, 0) -- node[below] {$\mathrm{W}^{+}$} (2,0) ;
   \draw[->] (0.5,-0.6)--(1.5,-0.6);
    \draw[style={electron}]
   (-1, -2) -- node[left] {p}(0, 0);
  \draw[style={electron}]
  (0, 0) -- node[left] {n}(-1, 2) ;
  \draw[style={electron}]
   (2, 0)  -- node[right] {e$^{+}$}(1, 2) ;
  \draw[style={electron}]
  (2, 0) -- node[right] {$\nu_{\mathrm{e}}$}(3, 2) ;
\end{tikzpicture} \end{minipage}\begin{minipage}[c]{0.1\textwidth}OR\end{minipage}
\begin{minipage}{0.4\textwidth}
\begin{tikzpicture}
\draw[style={boson}]
  (0, 0) -- node[below] {$\mathrm{W}^{+}$} (2,0) ;
   \draw[->] (0.5,-0.6)--(1.5,-0.6);
    \draw[style={electron}]
   (-1, -2) node[below=2.5pt] {u}-- (0, 0);
  \draw[style={electron}]
  (0, 0)  -- (-1, 2) node[above] {d};
  \draw[style={electron}]
   (-1.2, -2) node[below] {d}-- (-0.2, 0);
  \draw[style={electron}]
  (-0.2, 0)  -- (-1.2, 2) node[above] {d};
  \draw[style={electron}]
   (-1.4, -2) node[below=2.5pt] {u}-- (-0.4, 0);
  \draw[style={electron}]
  (-0.4, 0)  -- (-1.4, 2) node[above] {u};
  \draw[style={electron}]
   (2, 0)  -- node[right] {e$^{+}$}(1, 2) ;
  \draw[style={electron}]
  (2, 0) -- node[right] {$\nu_{\mathrm{e}}$}(3, 2) ;
\end{tikzpicture}  
\end{minipage}


\subsection{Electron capture}

An orbiting electron can be absorbed by a proton in the nucleus:
\[\mathrm{p}\mathrm{e}^{-}\longrightarrow\mathrm{n}\nu_{\mathrm{e}}\]

\begin{tikzpicture}
\draw[style={boson}]
 (0, 0) -- node[below] {$\mathrm{W}^{+}$} (2,0) ;
   \draw[->] (0.5,-0.6)--(1.5,-0.6);
      \draw[style={electron}]
   (-1, -2) -- node[left] {p} (0, 0);
  \draw[style={electron}]
  (0, 0) -- node[left] {n} (-1, 2) ;
  \draw[style={electron}]
   (3, -2)  -- node[right] {$\mathrm{e}^{-}$} (2, 0) ;
  \draw[style={electron}]
  (2, 0) -- node[right] {$\nu_{\mathrm{e}}$} (3, 2) ;
\end{tikzpicture}


\subsection{Neutrino--neutron collisions}
A neutron can absorb a neutrino, turning into a proton and electron:
\[\mathrm{n}\nu_{\mathrm{e}}\longrightarrow\mathrm{p}\mathrm{e}^{-}\]

\begin{tikzpicture}
\draw[style={boson}]
 (0, 0) -- node[below] {$\mathrm{W}^{+}$} (2,0) ;
   \draw[->] (1.5,-0.6)--(0.5,-0.6);
      \draw[style={electron}]
   (-1, -2) -- node[left] {n} (0, 0);
  \draw[style={electron}]
  (0, 0) -- node[left] {p} (-1, 2) ;
  \draw[style={electron}]
   (3, -2)  -- node[right] {$\nu_{\mathrm{e}}$} (2, 0) ;
  \draw[style={electron}]
  (2, 0) -- node[right] {$\mathrm{e}^{-}$} (3, 2) ;
\end{tikzpicture}

\subsection{Antineutrino--proton collisions}
A proton can absorb an anti electron neutrino, becoming a neutron and emitting a positron:
\[\mathrm{p}\nu_{\mathrm{e}}\longrightarrow\mathrm{n}\mathrm{e}^{+}\]

\begin{tikzpicture}
\draw[style={boson}]
 (0, 0) -- node[below] {$\mathrm{W}^{+}$} (2,0) ;
   \draw[->] (0.5,-0.6)--(1.5,-0.6);
      \draw[style={electron}]
   (-1, -2) -- node[left] {p} (0, 0);
  \draw[style={electron}]
  (0, 0) -- node[left] {n} (-1, 2) ;
  \draw[style={electron}]
   (3, -2)  -- node[right] {$\bar{\nu_{\mathrm{e}}}$} (2, 0) ;
  \draw[style={electron}]
  (2, 0) -- node[right] {$\mathrm{e}^{+}$} (3, 2) ;
\end{tikzpicture}

\subsection{Electron--proton collisions}

An electron can collide with a proton, emitting a neutron and a neutrino:
\[\mathrm{p}\mathrm{e}^{-}\longrightarrow\mathrm{n}\nu_{\mathrm{e}}\]

\begin{tikzpicture}
\draw[style={boson}]
 (0, 0) -- node[below] {$\mathrm{W}^{-}$} (2,0) ;
   \draw[->] (1.5,-0.6)--(0.5,-0.6);
      \draw[style={electron}]
   (-1, -2) -- node[left] {p} (0, 0);
  \draw[style={electron}]
  (0, 0) -- node[left] {n} (-1, 2) ;
  \draw[style={electron}]
   (3, -2)  -- node[right] {$\mathrm{e}^{-}$} (2, 0) ;
  \draw[style={electron}]
  (2, 0) -- node[right] {$\nu_{\mathrm{e}}$} (3, 2) ;
\end{tikzpicture}\\

All of the above interactions involve the weak interaction, and they have all been experimentally observed.

