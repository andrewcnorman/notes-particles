\section{Nuclear stability}

Within a nucleus, which normally has a diameter of about \SI{e-14}{m}, a large attraction force is needed to overcome the coulomb repulsion of the very closely packed protons.\footnote{The magnitude of this force is of the order \SI{e4}{N}.}

Experimental investigations of nuclear material show that it is very dense, and the density is independent of the nucleon number.  This implies that the force between nucleons is a short range force, only extending to the adjacent nucleons (if it extended further a density increase with $A$ would be expected, as more and more nucleons pulled tighter and tighter on each other!)

The nuclear force---called the strong force---is very complex and it is not possible to deduce its form precisely.  However, we can note that
\begin{itemize}
\item it is essential to keep the nucleus stable and bound together (balancing the effect of proton--proton electrostatic repulsion)
\item it is about \num{e8} times stronger than interatomic forces
\item it is charge independent, i.e.\  at any given separation the strong force between two neutrons is the same as that between two protons or between a proton and neutron
\item it is attractive down to about \SI{3}{fm}=\SI{3e-15}{m}
\item it has to be repulsive at very short range ($<0.5$~fm) otherwise there would be a tendency for the nucleus to collapse in on itself
\end{itemize}

\section{Radioactivity}

Radioactivity is the spontaneous disintegration of the nucleus of an atom, from which may be emitted some or all of the following 
\begin{itemize}
\item $\alpha$ alpha particles,
\item $\beta$ beta particles,
\item $\gamma$ gamma rays.
\end{itemize}

The process allows an unstable nucleus to become more stable, and its rate is not affected by chemical combinations or changes in physical environment.  Natural sources of background radiation (which must be taken into account when taking experimental radioactivity measurements) include cosmic rays, rocks (especially granite) and some luminous paints.

\subsection{$\alpha$ decay}
An $\alpha$ particle is identical to a helium nucleus, consisting of 2 protons and 2 neutrons.

When an element disintegrates by the emission of an $\alpha$ particle it turns into an element with chemical properties similar to those of an element two places earlier in the periodic table $(A,Z)\rightarrow(A-4,Z-2)$.

e.g.\ Radium  decays via $\alpha$ emission to form radon (Rn):
\[\isotope[226][88]{Ra}\longrightarrow\isotope[222][86]{Rn}+\isotope[4][2]{\alpha}.\]

\subsection{$\beta$ decay}
A $\beta$ particle is an electron. During $\beta$ decay a neutron changes into a proton.  When an element disintegrates by emission of a $\beta$ particle it turns into an element with chemical properties similar to an element one place later in the periodic table $(A,Z)\rightarrow(A,Z+1)$. 

e.g.\  Sodium-24 (also known as radiosodium) decays via $\beta$ emission to magnesium-24:
\[\isotope[24][11]{Na}\longrightarrow\isotope[24][12]{Mg}+\isotope[0][-1]{\beta}.\]

\subsection{$\gamma$ radiation}

Frequently, spare energy released during a radioactive disintegration is emitted as very penetrating and harmful $\gamma$ radiation (very short wavelength electromagnetic radiation).  Though the nucleus loses energy by $\gamma$ emission, the structure of the nucleus $(A,Z)$ does not change.
